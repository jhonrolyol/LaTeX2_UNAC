
% Preamble
\documentclass[12pt,a4paper]{book} % Libro
\usepackage[utf8]{inputenc} %
\usepackage{fontenc} %
\usepackage{xcolor} % Color
\usepackage{amsmath} %
\usepackage{amsfonts} %
\usepackage{amssymb} %
\usepackage[spanish]{babel} % Idioma


% Document
\begin{document}

  %% Chapter 1
    \chapter{Introducción al \LaTeX}
     
     %%% Section 1.1
        \section{Escritura}
             Existen:
         
         %%%% Subsection 1.1.1    
             \subsection*{Tipos de escritura}
         
              \begin{itemize}
                \item Si yo uso \texttt{textbf}\{\text{texto}\}, lo
                que este dentro se pondrá en negrita:
                \``\textbf{texto}''

                \item Si yo uso \texttt{textit}\{\text{texto}\}, lo
                que este dentro se pondrá en cursiva:
                \``\textit{texto}''
           
                \item Si yo uso \texttt{textsc}\{\text{texto}\}, lo
                que este dentro se pondrá todo en mayuscula
                \``\textsc{texto}''

                \item Si yo uso \texttt{textsf}\{\text{texto}\}, lo
                que este dentro se pondrá todo en mayuscula
                \``\textsf{texto}''
           
                \item Si yo uso \texttt{textsl}\{\text{texto}\}, lo
                que este dentro se pondrá todo en mayuscula
                \``\textsl{texto}''
              \end{itemize}

         %%%% Subsection 1.1.2
         \subsection*{Escritura de formulas}
           Existen tres formas de escribir formulas:\begin{itemize}
           \item \textbf{Escritura lineal}: Para ello solamente se
           deben utilizan un símbolo de dolar en cada extremo (\$).     
           
           \begin{center}
              Ella no te ama $m=\cfrac{y_2-y_1}{x_2-x_1}$, me fue
              infiel
          \end{center}

          \item \textbf{Escritura centrada}: Para ello se deben
          utilizan doble símbolo de dolar en cada extremo (\$\$).
          \\[0.2cm] No olvidar que la función de onda se escribe: $$
          \Psi(r,\theta,\phi)=R(r)\Theta (\theta,\phi)$$ La parte
          angular tiene solución con \textbf{armónicos esféricos}

          \item \textbf{Escritura enumerada}: Para ello debemos usar
          el comando \texttt{begin}\{\text{equation}\}\\[0.2cm]
          Según $\cdots$, la ecuación fundamental de la termodinámica
          es:\begin{equation}
          T\text{d}S=\text{d}U+p\text{d}V-\mu\text{d}N
          \end{equation}
          $$\text{d}S=\frac{\text{d}U}{T}+\frac{p\text{d}V}{T}
          \frac{\mu\text{d}N}{T}$$
          \end{itemize}
          
     %%% Section 1.2
         \section{Matrices}
         
         %%%% Subsection 1.2.1
             \subsection*{Vectores}
               $$\vec{v}=\textbf{v}=
               \begin{pmatrix}
                 a_{11} \\ a_{12} \\ a_{13} \\ \vdots \\ a_{1n}
               \end{pmatrix}$$
               $$\vec{v}=\textbf{v}=
               \begin{vmatrix}
                 a_{11} \\ a_{12} \\ a_{13} \\ \vdots \\ a_{1n}
              \end{vmatrix}$$

         %%%% Subsection 1.2.2
             \subsection*{Matrices}
               $$A=
               \begin{pmatrix}
                  a_{11} & a_{21} & a_{31} & \hdots & a_{m1} \\
                  a_{12} & a_{22} & a_{32} & \hdots & a_{m2} \\
                  a_{13} \\
                  \vdots \\
                  a_{1n}
              \end{pmatrix}$$


\end{document}





