\documentclass[12pt,a4paper]{article}
\usepackage[utf8]{inputenc}
\usepackage{fontenc}
\usepackage{xcolor}
\usepackage{amsmath}
\usepackage{amsfonts}
\usepackage{amssymb}
\usepackage[spanish]{babel}
\usepackage{anysize}
\usepackage{setspace}
\usepackage{hyperref}
\marginsize{3.50cm}{2.50cm}{3.00cm}{3.00cm}
\setlength{\parindent}{8mm}
\doublespace

\title{Taller de \LaTeX}
\author{Autor 1\and Autor 2}
\date{\today}

\begin{document}
\maketitle
\newpage

\tableofcontents
\newpage
\begin{abstract}
La mecánica cuántica a la rama de la física contemporánea dedicada al estudio de los objetos y fuerzas de muy pequeña escala espacial, es decir, de la materia a nivel del átomo y de las partículas que lo componen, así como los movimientos que las caracterizan.
\end{abstract}
\renewcommand{\abstractname}{Abstract}
\begin{abstract}
Quantum mechanics is the branch of contemporary physics dedicated to the study of objects and forces of very small spatial scale, that is, matter at the level of the atom and the particles that compose it, as well as the movements that characterize them.
\end{abstract}

\part{Introducción al \LaTeX}
\section{Escritura}\label{introduccion}
La teoría de la relatividad especial, también llamada teoria de la\linebreak relatividad restringida, es una teoría de la física publicada en 1905 por Albert Einstein. Surge de la observación de que la velocidad de la luz en el vacío es igual en todos los sistemas de referencia inerciales y de obtener todas las consecuencias del principio de {\it \large relatividad de Galileo}.
\section{Relatividad}
Según el, cualquier experimento realizado en un sistema de {\sc referencia inercial} se desarrollara de manera idéntica en cualquier otro sistema inercial. Información que aparecerá en la página.

Tengo 35 \$, $\o$
 
$$P=50\,\textsc{atm}$$

\begin{equation}
\label{Atmos}
P=50~\text{atm}
\end{equation}

\subsection{Tipos de escritura}
\subsubsection*{Párrafo}

\begin{itemize}
\item Si yo uso \texttt{textbf}\{\text{texto}\}, lo que este dentro se pondrá en negrita:\\``\textbf{texto}''

\item Si yo uso \texttt{textit}\{\text{texto}\}, lo que este dentro se pondrá en cursiva:\\``\textit{texto}''

\item Si yo uso \texttt{textsc}\{\text{texto}\}, lo que este dentro se pondrá todo en\linebreak mayuscula:\\``\textsc{texto}''

\item Si yo uso \texttt{textsf}\{\text{texto}\}, lo que este dentro se pondrá todo en\linebreak mayuscula:\\``\textsf{texto}''

\item Si yo uso \texttt{textsl}\{\text{texto}\}, lo que este dentro se pondrá todo~en~mayuscula~ss:\\``\textsl{texto}''

\end{itemize}

\subsection*{Escritura de formulas}\label{formulas}
Existen tres formas de escribir formulas:\begin{itemize}
\item \textbf{Escritura lineal}: Para ello solamente se deben utilizan un símbolo de dolar en cada extremo (\$).\begin{center}
Ella no te ama $m=\cfrac{y_2-y_1}{x_2-x_1}$, me fue infiel
\end{center}


\item \textbf{Escritura centrada}: Para ello se deben utilizan doble símbolo de dolar en cada extremo (\$\$).\\[0.2cm]
No olvidar que la función de onda se escribe: $$\Psi(r,\theta,\phi)=R(r)\Theta (\theta,\phi)$$ La parte angular tiene solución con \textbf{armónicos esféricos}

\item \textbf{Escritura enumerada}: Para ello debemos usar el comando \texttt{begin}\{\text{equation}\}\\[0.2cm]
Según $\cdots$, la ecuación fundamental de la termodinámica es:\begin{equation}
T\text{d}S=\text{d}U+p\text{d}V-\mu\text{d}N
\end{equation}
$$\text{d}S=\frac{\text{d}U}{T}+\frac{p\text{d}V}{T}-\frac{\mu\text{d}N}{T}$$
\end{itemize}

\section{Matrices}\label{matriz}
Segun De La Peña \cite{DLP} y Muñoz \cite{MJM}
\subsection*{Vectores}
$$\vec{v}=\textbf{v}=\begin{pmatrix}
a_{11} \\ a_{12} \\ a_{13} \\ \vdots \\ a_{1n}
\end{pmatrix}$$

$$\vec{v}=\textbf{v}=\begin{vmatrix}
a_{11} \\ a_{12} \\ a_{13} \\ \vdots \\ a_{1n}
\end{vmatrix}$$

\subsection*{Matrices}
$$A=\begin{pmatrix}
a_{11} & a_{21} & a_{31} & \hdots & a_{m1} \\
a_{12} & a_{22} & a_{32} & \hdots & a_{m2} \\
a_{13} \\
\vdots \\
a_{1n}
\end{pmatrix}$$

Como se pudo ver en la pagina \pageref{formulas} indicamos las formas es escribir las ecuaciones. Y la solución final la pudimos ver en la pagina \pageref{Atmos}. La lista de asistencia aparece \href{https://forms.gle/84VyN2QJq7twTHH87}{haciendo click aqui}

\renewcommand{\refname}{Bibliografía}
\begin{thebibliography}{2}
\bibitem{DLP} De La Peña, L. (2014). Introducción a la mecánica cuántica. Fondo de Cultura económica.
\bibitem{MJM} Muñoz, J. M. (2015). Mecánica cuántica y libre albedrío: cinco cuestiones fundamentales. Principia: an international journal of epistemology, 19(1), 65-92.
\end{thebibliography}



\end{document}
