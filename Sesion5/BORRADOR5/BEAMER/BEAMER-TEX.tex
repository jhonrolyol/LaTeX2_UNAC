\documentclass[12pt]{beamer}
\usetheme{Madrid}
\usepackage[utf8]{inputenc}
\usepackage{amsmath}
\usepackage{amsfonts}
\usepackage{amssymb}

\usepackage{sansmathaccent}
\pdfmapfile{+sansmathaccent.map}

\usepackage{fancyhdr}
\author{Fernando Flores Q.}
\title{\bf Taller de \LaTeX}

\usepackage{tikz}
\usepackage{multicol}
\usetikzlibrary{positioning}
\usepackage{listings}

%Personalización
\usefonttheme{serif}
\usecolortheme{albatross}

%\setbeamercovered{transparent} 
%\setbeamertemplate{navigation symbols}{} 
%\logo{} 
\institute{Facultad de Ciencias Naturales y Matemática} 
\date{Sáb. 5 de Noviembre del 2022} 
\subject{Callao} 
\begin{document}

\begin{frame}
\titlepage
\end{frame}

%\begin{frame}
%\tableofcontents
%\end{frame}

\begin{frame}{Introducción a Beamer}
Según $\cdots$, la ecuación fundamental de la termodinámica es:\begin{equation}
T\text{d}S=\text{d}U+p\text{d}V-\mu\text{d}N
\end{equation}
$$\text{d}S=\frac{\text{d}U}{T}+\frac{p\text{d}V}{T}-\frac{\mu\text{d}N}{T}$$
\end{frame}

\end{document}