\documentclass[12pt,a4paper]{article}
\usepackage[utf8]{inputenc}
\usepackage{amsmath}
\usepackage[T1]{fontenc}
\usepackage{amsfonts}
\usepackage{amssymb}
\usepackage[spanish,es-tabla]{babel}
\usepackage{makeidx}
\usepackage{hyperref}
\usepackage{graphicx}
\usepackage{subfig}
\usepackage{cite}
\usepackage{lmodern}
\usepackage{listings}
\usepackage{xcolor}
\usepackage{float}
\usepackage{xfrac}
\usepackage{fix-cm}
\usepackage{titlesec}
\usepackage{setspace}
\usepackage{fancyhdr}

\renewcommand{\thesection}{\Roman{section}}
\renewcommand{\thesubsection}{\arabic{section}.\arabic{subsection}}
\AtBeginDocument{\renewcommand{\contentsname}{ÍNDICE}}
\onehalfspace
\titleformat{\section}
{\bfseries\large}
{\large\thesection. }
{1pt}
{}

%Personalizacion del documento
\fancyhead[L]{}
\fancyhead[C]{}
\fancyhead[R]{}
\fancyfoot[L]{}
\fancyfoot[C]{}
\fancyfoot[R]{\thepage}
\renewcommand{\headrulewidth}{0pt}
\renewcommand{\footrulewidth}{0pt}

\graphicspath{{imagenes/}}
\usepackage[left=3.5cm,right=2.5cm,top=3cm,bottom=3cm]{geometry}

\setlength{\parindent}{0cm}

\begin{document}
\pagestyle{empty}
\begin{titlepage}
	\begin{center}
		{\textbf{UNIVERSIDAD NACIONAL DEL CALLAO}}\\
		{\textbf{FACULTAD DE CIENCIAS NATURALES Y MATEMÁTICA}}\\
		{\textbf{ESCUELA PROFESIONAL DE FÍSICA}}\\
		\vspace{2cm}
		\begin{figure}[h]
			\centering
			\includegraphics[height=7cm]{imagenes/logounac}
		\end{figure}
		\vspace{1cm}
		{\bf PROYECTO DE INVESTIGACIÓN}\\
		\vspace{2mm}
		\begin{center}
			{\textbf{``TITULO: DE LA TESIS''}}
		\end{center}

		\vspace{15mm}
		{NOMBRES Y APELLIDOS}\\
		\vspace{4mm}
		{LINEA DE INVESTIGACIÓN: INDICAR}\\

	\vspace{16mm}
	{CALLAO, 2022}\\
	\vspace{2mm}
	{PERÚ}
	\end{center}
\end{titlepage}

\newpage
\begin{center}
\bf \large INFORMACIÓN BÁSICA
\end{center}
\textbf{FACULTAD}: CIENCIAS NATURALES Y MATEMÁTICA\\[0.4cm]

\textbf{UNIDAD DE INVESTIGACIÓN}: FACULTAD DE CIENCIAS NATURALES Y MATEMÁTICA\\[0.4cm]

\textbf{TITULO}: DE LA TESIS\\[0.4cm]

\textbf{AUTOR}: NOMBRE Y APELLIDO / CODIGO ORCID / DNI\\[0.4cm]

\textbf{ASESOR}:\\[0.4cm]

\textbf{LUGAR DE EJECUCIÓN}: LUGAR.\\[0.4cm]

\textbf{UNIDADES DE ANÁLISIS}: UNIDADES.\\[0.4cm]

\textbf{TIPO DE INVESTIGACIÓN}: TIPO\\[0.4cm]

\textbf{LÍNEA DE INVESTIGACIÓN}: ÁREA\\[0.4cm]

\newpage
\setcounter{page}{1}
\pagestyle{fancy}
\tableofcontents

\newpage
%\listoffigures

\section*{INTRODUCCIÓN}
\addcontentsline{toc}{section}{INTRODUCCIÓN}
dsadas

\newpage
\section{PLANTEAMIENTO DEL PROBLEMA}
\subsection{Descripción de la realidad problemática}
sdadasdf

\newpage

\subsection{Formulación del problema}
\subsubsection{Problema General}
asfadfsad
\subsubsection{Problemas Específicos}
asfadfsad
\subsection{Objetivos}
\subsubsection{Objetivo General}
asfadfsad
\subsubsection{Objetivos Específicos}
asfadfsad

\newpage
\subsection{Justificación}
asfadfsad
\subsection{Delimitantes de la investigación}
asfadfsad

\newpage
\section{MARCO TEÓRICO}
\subsection{Antecedentes}
sdadasdf
\subsection{Bases teóricas}
asfadfsad
\subsection{Marco conceptual}
sdadasdf
\subsection{Definición de términos básicos}
asfadfsad

\newpage
\section{HIPÓTESIS Y VARIABLES}
\subsection{Hipótesis}
sdadasdf
\subsection{Operacionalización de variable}
asfadfsad


\newpage
\section{REFERENCIAS BIBLIOGRÁFICAS}
\bibliographystyle{apa}
\begin{thebibliography}{4}
\bibitem{C1} Cita


\end{thebibliography}
\end{document}
